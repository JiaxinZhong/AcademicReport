\documentclass{article}

\usepackage{AcademicReport}

% enable Chinese
% \usepackage[UTF8]{ctex}


% setting for levels
\usepackage{hyperref}
\setcounter{secnumdepth}{2}
\setcounter{tocdepth}{2}
\hypersetup{
    bookmarksopenlevel = 3,
    citecolor = red,
	colorlinks		=	true,
	bookmarks		=	true,
	bookmarksopen	=	true,
    % bookmarksdepth  =   2,
	pdfstartview	=	Fit,
	pdftitle		=	{Academic Report Template},
	pdfauthor		=	{Jiaxin Zhong},
}

% open book mark at arbitrary level
\usepackage[open, numbered]{bookmark}
% opt: numbered --- show numbers in the bookmark

% for some dummy text
\usepackage{lipsum}

% see https://tex.stackexchange.com/questions/226481/appendix-section-title
\usepackage[title]{appendix}


\title{\textbf{Academic Report Template}}
\author{Jiaxin Zhong}
\date{\today}

% \usepackage{mlmodern}


\begin{document}
\maketitle
% enable Page 1 of xx at the first page
\thispagestyle{firststyle}

\section{Introduction}
This is a \LaTeX{} template designed for daily writing the academic documents, e.g., progress reports, technical reports, and technical notes.

\lipsum[1]

\lipsum[2]

\lipsum[3]

\lipsum[4]

\lipsum[5]

\lipsum[6]

\section{Figures}
Figure~\ref{fig:39:f020390} shows the results of someting.
\begin{figure}[!htb]
    \centering
    \includegraphics[width = 0.4\textwidth]{example-image}
    \caption{Example figure}
    \label{fig:39:f020390}
\end{figure}


\section{Equations}
Equation~(\ref{eq:29jf2}) is an example of the labelled equation.
\begin{equation}
    f(x) = \int_0^x t\sin t \dd t
    \label{eq:29jf2}
\end{equation}

\section{Citations}
References: \cite{Zhong2020InsertionLossThin}



% Activate the appendix
% from now on sections are numerated with capital letters
\begin{appendices}
% \addcontentsline{toc}{section}{Appendices}

\section{Derivation of Eq.~(\ref{eq:29jf2})}
\lipsum[1]

\lipsum[1]

\lipsum[1]

\section{Illustration of Fig.~(\ref{fig:39:f020390})}
\lipsum[1]

\lipsum[1]
\end{appendices}


\addcontentsline{toc}{section}{References}
\bibliographystyle{plain}
\bibliography{zotero}

\end{document}


